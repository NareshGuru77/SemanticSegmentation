\documentclass[paper=a4,11pt,parskip=half,toc=listof]{scrartcl}
\usepackage{report_package}

\begin{document}
%%%%%%%%%%%%%%%%%%%%% Startseite %%%%%%%%%%%%%%%%%%%%%%%%%%%
\begin{titlepage}
\begin{minipage}[t]{0.5\textwidth}
\begin{Large}
    \begin{flushleft}
      \hspace{1cm} \makebox[3cm][c]{\includegraphics[height=8ex]{./logos/logo_hbrs.png} \vspace{1.8cm}}
    \end{flushleft}
\end{Large}
\end{minipage}

\vspace{0.07\textheight}
\begin{center}
 \begin{Large} \textbf{\ThesisUniversityCourse} \end{Large}\\
 \vspace{1em}
 \begin{Large} \textbf{-- \ThesisSemester --} \end{Large}\\
 \vspace{2em}
 \begin{Huge} \begin{spacing}{1.3} \textbf{\ThesisTitle} \end{spacing} \end{Huge}
 \vspace{2em}
 \begin{Large} \textbf{-- Report on dataset creation --} \end{Large}\\
 \vspace{2em}
  \begin{Large}\textbf{by} \end{Large}\\
 
 \vspace{2em}
 \begin{Large}\textbf{\ThesisAuthora}\end{Large}\\
 \begin{small}naresh.gurulingan@smail.inf.h-brs.de\end{small}\\
 \begin{small} Matr. no. 9030384\end{small}\\
\end{center}
\end{titlepage}

\newgeometry{top=4cm, bottom=3cm, left=4.5cm, right=3cm}

\newpage
\setcounter{page}{3} 
\begin{spacing}{1.14}
\tableofcontents
\end{spacing}

\clearpage{}
\listoftables % add list of tables
\clearpage{}
\listoffigures % add list of figures 
\clearpage{}
\include{acronym} % include the acronym thing
\clearpage{}

\setcounter{tocdepth}{4} 
\setcounter{secnumdepth}{4}
\setlength\parindent{0pt}  %indent disabled

\pagenumbering{arabic} % arabic numbers for the main part


\newpage
\section{Overview of the dataset}
Since semantic segmentation using deep learning is framed as a pixelwise classification task, an image of dimensions H$\times$W$\times$C requires a ground truth of dimensions H$\times$W, where H and W are the height and width of the image in the dataset having C number of channels. 

The scope of the dataset is to include objects associated to RoboCup @Work. The selected 18 objects are shown in \ref{Fig:1}.

\begin{figure}[!htb]
\centering
\begin{subfigure}{.3\textwidth}
  \centering
  \includegraphics[width=.5\linewidth]{axis}
  \caption{axis \cite{github_robocup@work}}
  \label{fig:axis}
\end{subfigure}%
\begin{subfigure}{.3\textwidth}
  \centering
  \includegraphics[width=.5\linewidth]{bearing}
  \caption{bearing \cite{github_robocup@work}}
  \label{fig:bearing}
\end{subfigure}
\begin{subfigure}{.3\textwidth}
  \centering
  \includegraphics[width=.5\linewidth]{bearingBoxAX01}
  \caption{bearing box AX01 \cite{github_robocup@work}}
  \label{fig:bearingBoxAX01}
\end{subfigure}\\
\vspace{3mm}
\begin{subfigure}{.3\textwidth}
  \centering
  \includegraphics[width=.5\linewidth]{bearingBoxAX16}
  \caption{bearing box AX16}
  \label{fig:bearingBoxAX16}
\end{subfigure}
\begin{subfigure}{.3\textwidth}
  \centering
  \includegraphics[width=.5\linewidth]{container_blue}
  \caption{container blue \cite{github_robocup@work}}
  \label{fig:container_blue}
\end{subfigure}
\begin{subfigure}{.3\textwidth}
  \centering
  \includegraphics[width=.5\linewidth]{container_red}
  \caption{container red \cite{github_robocup@work}}
  \label{fig:container_red}
\end{subfigure}\\
\vspace{3mm}
\begin{subfigure}{.3\textwidth}
  \centering
  \includegraphics[width=.5\linewidth]{distanceTube}
  \caption{distance tube \cite{github_robocup@work}}
  \label{fig:distanceTube}
\end{subfigure}
\begin{subfigure}{.3\textwidth}
  \centering
  \includegraphics[width=.5\linewidth]{F20_20_B}
  \caption{F20\_20\_B \cite{github_robocup@work}}
  \label{fig:F20_20_B}
\end{subfigure}
\begin{subfigure}{.3\textwidth}
  \centering
  \includegraphics[width=.5\linewidth]{F20_20_G}
  \caption{F20\_20\_G \cite{github_robocup@work}}
  \label{fig:F20_20_G}
\end{subfigure}\\
\vspace{3mm}
\begin{subfigure}{.3\textwidth}
  \centering
  \includegraphics[width=.5\linewidth]{M20}
  \caption{M20 \cite{github_robocup@work}}
  \label{fig:M20}
\end{subfigure}
\begin{subfigure}{.3\textwidth}
  \centering
  \includegraphics[width=.5\linewidth]{M20_100}
  \caption{M20\_100 \cite{github_robocup@work}}
  \label{fig:M20_100}
\end{subfigure}
\begin{subfigure}{.3\textwidth}
  \centering
  \includegraphics[width=.5\linewidth]{M30}
  \caption{M30 \cite{github_robocup@work}}
  \label{fig:M30}
\end{subfigure}\\
\vspace{3mm}
\begin{subfigure}{.3\textwidth}
  \centering
  \includegraphics[width=.5\linewidth]{motor}
  \caption{motor \cite{github_robocup@work}}
  \label{fig:motor}
\end{subfigure}
\begin{subfigure}{.3\textwidth}
  \centering
  \includegraphics[width=.5\linewidth]{R20}
  \caption{R20 \cite{github_robocup@work}}
  \label{fig:R20}
\end{subfigure}
\begin{subfigure}{.3\textwidth}
  \centering
  \includegraphics[width=.5\linewidth]{S40_40_B}
  \caption{S40\_40\_B \cite{github_robocup@work}}
  \label{fig:S40_40_B}
\end{subfigure}\\
\vspace{3mm}
\begin{subfigure}{.3\textwidth}
  \centering
  \includegraphics[width=.5\linewidth]{S40_40_G}
  \caption{S40\_40\_G \cite{github_robocup@work}}
  \label{fig:S40_40_G}
\end{subfigure}
\begin{subfigure}{.3\textwidth}
  \centering
  \includegraphics[width=.5\linewidth]{em_01}
  \caption{em\_01}
  \label{fig:em_01}
\end{subfigure}
\begin{subfigure}{.3\textwidth}
  \centering
  \includegraphics[width=.5\linewidth]{em_02}
  \caption{em\_02}
  \label{fig:em_02}
\end{subfigure}
\caption{Different objects required in the dataset}
\label{Fig:1}
\end{figure}

Each of the objects were taken individually, placed on 3 different backgrounds and 30 images were taken. This lead to a total of 540 images which were to be manually labeled. Since, every pixel of the images needs to be labeled, the process of manual annotation would be time consuming. Therefore, a decision was made to first annotate the 540 images and later decide whether more images could be taken based on the effort required for annotation.

\section{Selection of a labeling tool}
In order to reduce the time required to annotate an image, it was imperative to select a tool which is specifically designed for semantic segmentation and also provides algorithms which helps the annotator by providing labeling automation to the highest possible extent.

The following available tools were evaluated for ease of use and time taken for annotation:
	\begin{itemize}
		\item LabelMe: web based tool is public and data would also be public.
		\item LabelMe Matlab toolbox: yet to try..
		\item University bonn annotation tool:
		\item Pixel annotation tool (using watershed algorithm): works in windows. Seems to be useful.
		\item Ratsnake: tool dint seem to be useful although the website had options like superpixel suggestions.
		\item LabelImg: Can be used but time consuming.
		\item Figi: used in medical image segmentation. Has many options. Still exploring.
		\item Supervisely.
		\item MATLAB ImageLabeler available in release R2017b (Computer Vision Toolbox).
	\end{itemize}

\section{Description of the labeling process}
\label{section:process}
MATLAB ImageLabeler was used for the labeling process. At first, label definitions are created and exported to a .mat file. This file is used to load label definitions for all images to maintain consistency of labels. The contents of the .mat file is shown in the figure\ref{Fig:2}.

	\begin{figure}[htb!]
		\centering
		\includegraphics[scale=0.7]{labelDef}
		\caption{Contents of the labelDefs .mat file}
		\label{Fig:2}
	\end{figure}
	
The ImageLabeler app, by default, provides different tools which help create pixelwise labels\ref{Fig:3}. These tools become accessible once an image and the label definitions are loaded. A short description of the tools is given below:
	\begin{figure}[htb!]
		\centering
		\includegraphics[scale=0.55]{label_tools}
		\caption{Tools provided by the ImageLabeler app}
		\label{Fig:3}
	\end{figure}
	
	\begin{itemize}
		\item Polygon: This can be used to trace an object boundary by placing dots. Once a closed contour is created, pixels within the contour get assigned the corresponding object label.
		\item Smart Polygon: Can be used in a similar fashion like the Polygon tool. This tool, in addition, tries to reach out to the nearby edges of the drawn polygon.
		\item Brush and Erase: Square shaped brush and eraser to either label a region or remove labels from a region. The size of the square can be changed by using the Brush Size slider.
		\item Flood Fill: This tool provides same labels to pixels which are similar in terms of the intensity with the selected pixel.
		\item Label Opacity: This tool provides a sliding bar which varies the opacity of the overlayed labels on the image. This is helpful to visualize the assigned labels.
		\item Zoom In, Zoom Out, Pan: These tools improve the ease of labeling by providing means to focus on particular regions by zooming and panning.
	\end{itemize}
	
The ImageLabeler app by default assigns different colors to different objects to aid visualization. The label colors are shown in the ROI Label Definition window\ref{Fig:4}.
	\begin{figure}[htb!]
		\centering
		\includegraphics[scale=0.6]{roi_label_defintions}
		\caption{ROI Label Definitions window}
		\label{Fig:4}
	\end{figure}
	
The ImageLabeler app does not provide any tool to label all unlabeled pixels as background. In order to save time, the following workarounds have been used:
	\begin{itemize}
		\item The images taken for the dataset each have only one object in them.
		\item Only the object region is labeled.
		\item Since the ImageLabeler app does not provide any tool to label all unlabeled pixels as background, a python code which simply reads the label image and replaces unlabeled values 0 with background label value 19, was used for this purpose. The code is also used to double check the label image in order to avoid noisy labeling.
	\end{itemize}
	
The Export Labels -> To File option can be used to save the annotations. This is done for all images individually to arrive at the folder structure shown in \ref{Fig:5a}.
	
The saved .mat file can be loaded into ImageLabeler again to further modify labels if required later. The 'Label\_1.png' file located in the PixelLabelData folder (as can be seen in \ref{Fig:5a}) is the label image. This image is renamed to have the same name as the image file and a folder structure as in \ref{Fig:5b} is created by using a python code.
	
\begin{center}
	\begin{figure}[!htb]
		\begin{subfigure}{.5\textwidth}
			\centering
			\includegraphics[width=1\linewidth]{folder_structure}
			\caption{Folder structure of saved labels}
			\label{Fig:5a}
		\end{subfigure}
		\begin{subfigure}{.5\textwidth}
			\centering
			\includegraphics[width=1\linewidth]{folder_structure_aug}
			\caption{Rearranged folder structure}
			\label{Fig:5b}
		\end{subfigure}
		\caption{Different folder structures}
		\label{Fig:5}
	\end{figure}
\end{center}

The final folder structure is shown in \ref{Fig:6}. The image folder and label folder are similar and contain object images and corresponding label images with same names.

\begin{center}
	\begin{figure}[!htb]
		\begin{subfigure}{.5\textwidth}
			\centering
			\includegraphics[width=1\linewidth]{folder_image}
			%\caption{}
			\label{Fig:6a}
		\end{subfigure}
		\begin{subfigure}{.5\textwidth}
			\centering
			\includegraphics[width=1\linewidth]{folder_label}
			%\caption{}
			\label{Fig:6b}
		\end{subfigure}
		\caption{Folder structure showing different object folders in both image and label folders.}
		\label{Fig:6}
	\end{figure}
\end{center}

\section{About the artificial image generation algorithm}
\subsection{Motivation}
	\begin{itemize}
		\item Manually labeling 540 images with the described process in \ref{section:process} takes roughly 2160 minutes (roughly 4 minutes per image). This is equivalent to around 4 working days. Hence, creating a large dataset with manual labeling is not feasible.
		\item Taking images in a variety of real world backgrounds is also time consuming.
		\item Labeling images with multiple objects would take an even longer time.
	\end{itemize}
	
These drawbacks could be overcome by randomly placing objects on a variety of different background images automatically using an algorithm.

\subsection{Working}
The entire process of artificial image generation can be divided into 6 broad steps:
	\begin{itemize}
		\item \textbf{External interface}: An interface to obtain possible parameters to control the generation process.
		\item \textbf{Get backgrounds and data}: Fetch all the backgrouds, images and corresponding labels from the provided respective background, image and label paths.
		\item \textbf{Get object details}: Fetch details regarding every object and its different scales. The details include information regarding object locations in an image, object values, label values, the object name, points in pixel space denoting a bounding rectangle around the object, and object area.
		\item \textbf{Generate augmenter list}: Every element in the augmenter list denotes an artificial image and contains information including the chosen background image, the number of objects to place in the artificial image, which objects from the object details list are selected and locations in pixel space where the selected objects need to be placed.
		\item \textbf{Generate artificial images}: Based on every element in the augmenter list, artificial images are generated and corresponding labels are generated.
		\item \textbf{Generate preview}: The generated images and labels are visualized to verify the generation process.
	\end{itemize}
		

	
\subsection{Generator options}

A number of arguments can be configured to control the generation process. Configuration of generator options is possible through command line GUI. Details regarding the arguments are provided in \ref{Table:1} and \ref{Table:2}. 

\begin{table}[!htb]
\centering
\begin{tabular}{|c|c|c|c|c|c|c|c|}
\hline 
\textbf{Generator options} & Description \\ 
\hline 
\textbf{image\_dimension} & \makecell{Dimension of the real images.} \\ 
\hline 
\textbf{num\_scales} & \makecell{Number of scales including original object scale.} \\ 
\hline 
\textbf{backgrounds\_path} & \makecell{Path to directory where the background images are located.} \\ 
\hline 
\textbf{image\_path} & \makecell{Path to directory where real images are located.} \\ 
\hline 
\textbf{label\_path} & \makecell{Path to directory where labels are located.} \\ 
\hline 
\textbf{real\_img\_type} & \makecell{The format of the real image.} \\ 
\hline 
\textbf{min\_obj\_area} & \makecell{Minimum area in percentage allowed for \\an object in image space.} \\ 
\hline 
\textbf{max\_obj\_area} & \makecell{Maximum area in percentage allowed for \\an object in image space.} \\ 
\hline 
\textbf{save\_label\_preview} & \makecell{Save image+label in single image for preview.} \\ 
\hline 
\textbf{save\_obj\_det\_label} & \makecell{Save object detection labels in csv files.} \\ 
\hline  
\textbf{save\_mask} & \makecell{Save images showing the segmentation mask.} \\ 
\hline 
\textbf{image\_save\_path} & \makecell{Path where the generated artificial image needs\\ to be saved.} \\ 
\hline 
\textbf{label\_save\_path} & \makecell{Path where the generated segmentation label needs\\ to be saved.} \\ 
\hline 
\textbf{preview\_save\_path} & \makecell{Path where object detection labels needs to be saved.} \\ 
\hline 
\textbf{obj\_det\_save\_path} & \makecell{Path where object detection labels needs to be saved.} \\ 
\hline 
\textbf{mask\_save\_path} & \makecell{Path where segmentation masks needs to be saved.} \\ 
\hline 
\textbf{start\_index} & \makecell{Index from which image and label names should start.} \\ 
\hline 
\textbf{name\_format} & \makecell{The format for image file names.} \\
\hline 
\textbf{remove\_clutter} & \makecell{Remove images cluttered with objects.} \\
\hline 
\textbf{num\_images} & \makecell{Number of artificial images to generate.} \\ 
\hline 
\textbf{max\_objects} & \makecell{Maximum number of objects allowed in an image.} \\ 
\hline 
\textbf{num\_regenerate} & \makecell{Number of regeneration attempts of removed details dict.} \\ 
\hline 
\textbf{min\_distance} & \makecell{Minimum pixel distance required between two objects.} \\ 
\hline 
\textbf{max\_occupied\_area} & \makecell{Maximum object occupancy area allowed.} \\ 
\hline 
\textbf{scale\_ranges} & \makecell{Can be used to change the zoom range of specific objects.} \\ 
\hline 
\end{tabular}
\caption{Description of generator options} 
\label{Table:1}
\end{table}

\begin{table}[!htb]
\centering
\begin{tabular}{|c|c|c|c|c|c|c|c|}
\hline 
\textbf{Generator options} & Default value & \makecell{Is required?} \\ 
\hline 
\textbf{image\_dimension} & \makecell{[480, 640]} & \makecell{Not required} \\ 
\hline 
\textbf{num\_scales} & \makecell{'randomize'} & \makecell{Not required} \\ 
\hline 
\textbf{backgrounds\_path} & \makecell{-} & \makecell{Required} \\ 
\hline 
\textbf{image\_path} & \makecell{-} & \makecell{Required} \\ 
\hline 
\textbf{label\_path} & \makecell{-} & \makecell{Required} \\ 
\hline 
\textbf{real\_img\_type} & \makecell{'.jpg'} & \makecell{Not required} \\ 
\hline 
\textbf{min\_obj\_area} & \makecell{20} & \makecell{Not required} \\ 
\hline 
\textbf{max\_obj\_area} & \makecell{70} & \makecell{Not required} \\ 
\hline 
\textbf{save\_label\_preview} & \makecell{False} & \makecell{Not required} \\ 
\hline 
\textbf{save\_obj\_det\_label} & \makecell{False} & \makecell{Not required} \\ 
\hline  
\textbf{save\_mask} & \makecell{False} & \makecell{Not required} \\ 
\hline 
\textbf{image\_save\_path} & \makecell{-} & \makecell{Required} \\ 
\hline 
\textbf{label\_save\_path} & \makecell{-} & \makecell{Required} \\ 
\hline 
\textbf{preview\_save\_path} & \makecell{None} & \makecell{Required if \\save\_label\_preview is True} \\ 
\hline 
\textbf{obj\_det\_save\_path} & \makecell{None} & \makecell{Required if \\save\_obj\_det\_label is True} \\ 
\hline 
\textbf{mask\_save\_path} & \makecell{None} & \makecell{Required if \\save\_mask is True} \\ 
\hline 
\textbf{start\_index} & \makecell{0} & \makecell{Not required} \\ 
\hline 
\textbf{name\_format} & \makecell{'\%05d'} & \makecell{Not required} \\
\hline 
\textbf{remove\_clutter} & \makecell{True} & \makecell{Not required} \\
\hline 
\textbf{num\_images} & \makecell{20} & \makecell{Not required} \\ 
\hline 
\textbf{max\_objects} & \makecell{10} & \makecell{Not required} \\ 
\hline 
\textbf{num\_regenerate} & \makecell{100} & \makecell{Not required} \\ 
\hline 
\textbf{min\_distance} & \makecell{100} & \makecell{Not required} \\ 
\hline 
\textbf{max\_occupied\_area} & \makecell{0.8} & \makecell{Not required} \\ 
\hline 
\textbf{scale\_ranges} & \makecell{None} & \makecell{Not required} \\ 
\hline 
\end{tabular}
\caption{Default value of generator options and whether the options are required to be set.}
\label{Table:2}
\end{table}

\subsection{Sample results}
Sample results of the artificial image generation algorithm can be seen in \ref{Fig:8}. The bounding box represents the object detection label and the different colors of the segmentation labels denote different label values.

	\begin{figure}[htb!]
		\centering
		\includegraphics[scale=0.3]{sample_result_1}
		\includegraphics[scale=0.3]{sample_result_2}
		\includegraphics[scale=0.3]{sample_result_3}
		\caption{Sample results produced by the artificial image generation algorithm. In each row, the image on the left shows the generated artificial image and the image on the right shows a visualization of the semantic segmentation label and object detection label. At the top of every label visualization image, the objects in the image and their corresponding colors in the visualization are indicated.}
		\label{Fig:8}
	\end{figure}

\section{Dataset variants:}
Different variants of the dataset are created based on the properties of the objects in the dataset, and the type of  background images used for generation of artificial images.
	\subsection{Motivation}
		Looking into the objects present in the dataset, it is apparant that some objects are similar in certain aspects. For instance, the objects m20 and m30 are very similar to each other except that m30 is bigger in size and has a slightly different color. Because of the similarities existing among objects, the segmentation model could face certain difficulties as listed below:
		\begin{itemize}
			\item Inability to distinguish size: The segmentation model is given no information regarding the positions of the camera or the object in the real world. If camera extrinsic calibration information is available to the segmentation model, the model could possibly learn to distinguish different sizes. However, such information is not available. In addition, the objects in the artificial images are randomly scaled to different sizes thereby removing any size related information available.
			\item Inability to distinguish subtle variations in color: The real images were taken under different lighting conditions. As a result, there is no consistent difference in color information available between classes. This makes it difficult for the segmentation model to learn patterns in color information.
			\item Inability to distinguish shapes: Certain objects are closly related to each other in terms of shape and differ only slightly. For instance, bearing\_box\_ax16 and bearing\_box\_ax01 are similar in shape except in a few viewpoints as illustrated in Fig. In such cases, in certain viewpoints, the segmentation model would not be able to distinguish between similarly shaped objects.
			
			\begin{figure}[htb!]
    			\centering
    			\begin{subfigure}{0.5\textwidth}
        			\centering
        			\includegraphics[scale=0.3]{ax01_diff}
        			\caption{Lorem ipsum}
    			\end{subfigure}
    			~ 
    			\begin{subfigure}{0.5\textwidth}
        			\centering
        			\includegraphics[scale=0.4]{ax16_diff}
        			\caption{Lorem ipsum, lorem ipsum,Lorem ipsum, lorem ipsum,Lorem ipsum}
    			\end{subfigure}
    			\caption{Caption place holder}
			\end{figure}
	\end{itemize}

\section{Meta-data of the dataset}

Details regarding the dataset is provided in table \ref{Table:3}.

\begin{table}[!htb]
\centering
\begin{tabular}{|c|c|c|c|c|c|c|c|}
\hline 
    & Training & Validation & Test \\ 
\hline 
Real Images & \makecell{30 per object.\\ Total: 30$\times$18=540} & \makecell{5 per object.\\ Total: 5$\times$18=90} & \makecell{5 per object.\\ Total: 5$\times$18=90} \\ 
\hline 
Augmented Images & 5000 & 810 & 810 \\ 
\hline 
Total Images & 5540 & 900 & 900 \\ 
\hline 
\end{tabular}
\caption{Meta-data of the dataset} 
\label{Table:3}
\end{table}

\section{Notable features of the artificial image generator}

\section{Conclusion and possible directions of improvement}

Creating a custom dataset for a desired application is evidently challenging. To overcome the time consuming nature of creating annotations for semantic segmentation, choices such as 1. placing just 1 object per image while taking real images and 2. augmenting the objects on a random selection of diverse backgrounds, were made. This method of augmentation, although inspired by dataset generation method used in \cite{DBLP:journals/corr/abs-1709-00849} and the Synthia dataset \cite{RosCVPR16}, takes a different approach. Unlike \cite{DBLP:journals/corr/abs-1709-00849}, which uses 3D CAD models, this approach does not require any 3D models. Also, this approach does not require a virtual world as used by the Synthia dataset \cite{RosCVPR16}. However, the effectiveness of the dataset is yet to be verified by training and testing available segmentation models.
The following list provides possible directions of improvement:
	\begin{itemize}
		\item The ImageLabeler app by default saves the label '.png' file with the name 'Label\_1.png' in a folder called PixelLabelData. A automation script can be written and added to the ImageLabeler to provide options to save the label file in a way the user wants.
		\item Creating a way to replace all unlabeled pixels with the label value of 'background' from within the ImageLabeler would be helpful. For now, this is done by first exporting the label, then loading the label using opencv in python to replace 0 (value of unlabeled pixels) with 19 (value of 'background').
		\item The augmentation script is written in python and is independent of the MATLAB ImageLabeler app. This can be improved by including a way to start augmentation right from the ImageLabeler.
		\item A GUI for the augmentation script can be created to improve the ease of use.
	\end{itemize}

\newpage
\addcontentsline{toc}{section}{References}

\bibliographystyle{plain}
\bibliography{references}
\end{document}
