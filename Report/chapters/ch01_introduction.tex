%!TEX root = ../report.tex

\chapter{Introduction}

In recent years, deep learning has significantly impacted research in the field of computer vision. Variations of Convolutional Neural Network architectures have shown state-of-the-art performance in computer vision tasks such as image classification \cite{DBLP:journals/corr/HeZRS15}, object detection \cite{DBLP:journals/corr/RedmonDGF15}, action recognition \cite{DBLP:journals/corr/SimonyanZ14} and semantic segmentation \cite{DBLP:journals/corr/abs-1802-02611}. A considerable part of this success comes from the supervised learning paradigm through which the networks are trained with labeled samples.

State-of-the-art deep learning techniques in semantic segmentation also make use of the supervised learning paradigm. Semantic segmentation is treated as a pixelwise classification problem with the goal of assigning a class from a list of desired classes to every pixel in an image. The resultant image splits objects of interest into different regions thereby achieving the intended segmentation in a meaningful manner.

\section{Motivation}

Semantic segmentation is a rich source of information

\subsection{...}

%\lipsum[6-10]

\subsection{...}


\section{Challenges and Difficulties}
\subsection{...}

%\lipsum[11-15]

\subsection{...}

\subsection{...}



\section{Problem Statement}
\subsection{...}

%\lipsum[21-30]

\subsection{...}


\subsection{...}
