%!TEX root = ../report.tex

\chapter{DCNN and Semantic Segmentation}

	In this chapter, we look into the basic concepts of neural networks and later, how such concepts have been used for the task of semantic segmentation. In section  \textbf{[todo]}

\section{Artificial Neural Networks}

Artificial Neural Networks (ANN), inspired by the neural networks in our brain, was designed to learn tasks without explicitly programming descriptive features of the concerned tasks. An ANN is made up of processing units called neurons which performs a non-linear transformation, using an activation function, of the weighted linear combination of inputs.  Many such neurons are connected to one another in an ANN resulting in its ability to learn highly non-linear function mappings from input to output space. 


\subsection{Multilayer Perceptron}



\section{Convolutional Neural Networks}

\subsection{Covolution theorem}

\subsection{CNN Architecture}

\subsubsection{Convolutional layer}

\subsubsection{Pooling layer}

\subsubsection{Activation layer}

\subsubsection{Batch normalization layer}

\subsubsection{Fully-connected layer}

\section{Deep Learning for Semantic Segmentation}

