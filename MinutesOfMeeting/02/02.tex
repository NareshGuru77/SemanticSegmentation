\documentclass[14pt]{extarticle}
\usepackage{graphicx}
\usepackage{hyperref} 
\usepackage{enumitem}
\usepackage{makecell}
\usepackage{amsmath}
\usepackage{amssymb}
\usepackage{gensymb}
\usepackage[T1]{fontenc}
\usepackage{fancyhdr}
%\usepackage[shortlabels]{enumerate}

\graphicspath{ {/home/nareshguru77/Documents/learning_materials/Semester_2/RandD/Semantic_Segmentation/first_report} }

\usepackage[letterpaper, portrait, margin=0.8in]{geometry}

\pagestyle{fancy}
\fancyhf{}
%\rhead{Share\LaTeX}
\lhead{Embedded Semantic Segmentation}
\rfoot{Page \thepage}
\lfoot{ HBRS [MAS WS17-18]}

\begin{document}

\begin{center}
\begin{Large}

\underline{\textbf{Second meeting}}

\end{Large}
\end{center}

\textbf{Going through conventional image segmentation methods:}

\begin{enumerate}

\item K-means clustering

\item To read: \href{https://arxiv.org/pdf/1602.06541.pdf}{A Survey of Semantic Segmentation}

\end{enumerate}

\textbf{Toy neural network problems:}

\begin{enumerate}

\item 3 and 4 layer neural network for classification: Odd number or not, Prime number or not.
\item 3 and 4 layer neural network to learn 1 bit sum output of 8 bits.

\end{enumerate}

\textbf{Trying to run tensorflow tutorial on convolutional neural networks:}
\begin{itemize}

\item Tutorial link: \href{https://www.tensorflow.org/tutorials/layers}{weblink}

\end{itemize}

\textbf{Running an implementation of FCN and understanding the concepts:}
\begin{enumerate}

\item implementation from: \href{https://github.com/MarvinTeichmann/tensorflow-fcn}{Weblink}

\end{enumerate}

\begin{thebibliography}{9}

\bibitem{1}
\href{http://vision.stanford.edu/teaching/cs231b_spring1213/slides/segmentation.pdf}{StanfordVision - Segmentation}

\bibitem{2}

 
\bibitem{3}


\end{thebibliography}

\end{document}