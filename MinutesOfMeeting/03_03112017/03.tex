\documentclass[14pt]{extarticle}
\usepackage{graphicx}
\usepackage{hyperref} 
\usepackage{enumitem}
\usepackage{makecell}
\usepackage{amsmath}
\usepackage{amssymb}
\usepackage{gensymb}
\usepackage[T1]{fontenc}
\usepackage{fancyhdr}
%\usepackage[shortlabels]{enumerate}

\graphicspath{ {/home/nareshguru77/Documents/learning_materials/Semester_2/RandD/Semantic_Segmentation/first_report} }

\usepackage[letterpaper, portrait, margin=0.8in]{geometry}

\pagestyle{fancy}
\fancyhf{}
%\rhead{Share\LaTeX}
\lhead{Embedded Semantic Segmentation}
\rfoot{Page \thepage}
\lfoot{ HBRS [MAS WS17-18]}


\begin{document}

\begin{center}
\begin{Large}

\underline{\textbf{Third meeting}}\\
\vspace{3mm}
\today

\end{Large}
\end{center}

\textbf{Graph based image segmentation}

\begin{enumerate}
	\item Initial results in Efficient Graph based image segmentation.
	\item The image is first smoothed with a gaussian filter to remove noise.
	\item Every pixel is considered as a vertex and is connected to 4/8 neighbour pixels through edges.
	\item Each edge is assigned a weight. In the implementation the distance between [r,g,b] vector of connected vertices is taken as weights.
	\item The vertices are then sorted and build in the form of a minimum spanning tree.
	\item The tree is then cut into several components and components with a minimum component size is taken as a cluster and given a color.
	\item Yet to completely understand the method..

\end{enumerate}

\textbf{Current state-of-the-art for ASW:}
\begin{enumerate}
	\item Methods focusing on accuracy such as \cite{1} seems to be deep architectures which may not be suitable for embedded implementations.
	\item There are papers which focus on deep learning for embedded applications such as \cite{2}\cite{3}.
	\item Which direction to focus on first?
\end{enumerate}

\textbf{Initial problem formulation for ASW:}
\begin{itemize}
	\item Robocup specific dataset generation.
	\item Lean model (<1 s and 10MB).
	\item implementing and integrating the model to the robocup environment.
	
\end{itemize}

\textbf{Notes:}
\begin{itemize}
	\item Dataset generation along with Debaraj.
	\item 
\end{itemize}

\begin{thebibliography}{9}

\bibitem{1}
Liang-Chieh Chen, George Papandreou, Iasonas Kokkinos, Kevin Murphy, Alan L. Yuille, " DeepLab: Semantic Image Segmentation with Deep Convolutional Nets, Atrous Convolution, and Fully Connected CRFs", arXiv preprint, 2016. \href{https://arxiv.org/abs/1606.00915}{Weblink}

\bibitem{2}
Adam Paszke, Abhishek Chaurasia, Sangpil Kim, Eugenio Culurciello, " ENet: A Deep Neural Network Architecture for Real-Time Semantic Segmentation", arXiv preprint, 2016. \href{https://arxiv.org/abs/1606.02147}{Weblink}

\bibitem{3}
Andrew G. Howard, Menglong Zhu, Bo Chen, Dmitry Kalenichenko, Weijun Wang, Tobias Weyand, Marco Andreetto, Hartwig Adam, " MobileNets: Efficient Convolutional Neural Networks for Mobile Vision Applications ", arXiv preprint, 2017. \href{https://arxiv.org/abs/1704.04861}{Weblink}
 
\bibitem{4}
Github link with paper collection for embedded/mobile neural nets \href{https://github.com/csarron/emdl}{Github}

\end{thebibliography}

\end{document}